
\section{FEAST file format}

Extension: {\bf .feast}  

   
\subsubsection*{Preliminaries}
FEAST files are plain ASCII files in a line-based format, using shared-vertex data representations. Empty lines are not allowed. 
Comment lines can be arbitrarily placed, the special character~\# marks any such line. If the comment 
character~\# occurs in a line, the rest of this line will be ignored by FEAST.\\

Each FEAST \emph{project} consists of six separate files:
\begin{enumerate}
\item header file (filename.feast)
\item static geometry file (filename.fgeo)
\item fictitious boundary file (filename.ffgeo)
\item mesh file (filename.fmesh)
\item partitioning file (filename.fpart)
\item boundary condition file (filename.fbc)
\end{enumerate}


%%%%%%%%%%%%%%%%%%%%%%%%%%%%%%%%%%%%%%%%%%%%%%%%%%%%%%%%%%%%%%%%%%%%%%%%%%%%%%%


\subsection*{Header file}

The header file lists those files that are included in the current project.

\begin{code}{FEAST 2D -- Header}{Spezifikation_feast_2d_header}
\begin{verbatim}       
FEAST              # ID-tag
2D                 # mode
3                  # version major
0                  # version minor
description        # single line description of the project
# file locations
filename.fgeo      # relative path to static geometry file
filename.ffgeo     # relative path to fictitious geometry
filename.fmesh     # relative path to mesh file
filename.fpart     # relative path to partitioning file
filename.fbc       # relative path to boundary condition file
\end{verbatim}
\end{code}

The keyword \texttt{NONE} instead of a path indicates that no such file is used. Note that FEAST and the DeViSoR might crash if you exclude 
vital files.

Paths are relative to the location of the header file. 


%%%%%%%%%%%%%%%%%%%%%%%%%%%%%%%%%%%%%%%%%%%%%%%%%%%%%%%%%%%%%%%%%%%%%%%%%%%%%%


\subsection*{Geometry files}

Both static and fictitious geometry files have the same format.

Each geometry file starts with the following header:

\begin{code}{FEAST 2D -- geometry}{Spezifikation_feast_2d_numbers1}
\begin{verbatim}
FEAST_GEOM         # ID-tag
2D                 # mode
3                  # version major
0                  # version minor
6  # number of boundary components
\end{verbatim}
\end{code}

Supported boundary types are: 
\begin{itemize}
\item segment lists containing segments (single type or mixed) of the following types
\begin{itemize}
\item line segments defined by cartesian start coordinates and vector from start to end point
\item circle segments defined by midpoint cartesian coordinates, radius plus dummy, start and end angle in radians 
(angles are measured counterclockwise starting from the 3 o'clock position)
\item NURBS segments are defined by the degree $n$, a flag to distinguish between open and closed curves, $m+1$ control 
points, $m+1$ weights and a node vector of length $m+1$.
\end{itemize}
For each segment, the parameter interval is stored additionally, including the starting point, excluding the endpoint.
\item analytic (via reference to some object file which returns cartesian coordinates for each parameter value and maximum parameter value)
\end{itemize}

All boundary types are editable in the \devisor, for analytic boundaries, this
means editing and recompiling some Fortran or C or whatever code. These
functions have to provide three calls: \texttt{PARXC(t)}, \texttt{PARYC(t)},
\texttt{TMAXC} returning $x$ and $y$ coordinates for a given parameter value
and the length of the parameter interval (implicitly starting at parameter
value $0$).


Note that closed circles and closed NURBS curves must be the only boundary
type in each segment list!

\begin{code}{FEAST 2D -- Boundary components}{Spezifikation_feast_2d_rand}
\begin{verbatim}       
4         # number of segments on first boundary
0         # type of first segment: line
-1.0 1.0  # cartesian coordinates of start point
0.0 -2.0  # vector from start to end point
0.0 0.25  # parameter interval
0         # type of second segment: line
-1.0 -1.0 # cartesian coordinates of start point
2.0 0.0   # vector from start to end point
0.25 0.5  # parameter interval
0         # type of third segment: line
1.0 -1.0  # cartesian coordinates of start point
0.0 2.0   # vector from start to end point
0.5 0.75  # parameter interval
0         # type of 4th segment: line
1.0 1.0   # cartesian coordinates of start point
-2.0 0.0  # vector from start to end point
0.75 1.0  # parameter interval

1         # number of segments on second boundary
1         # type of first (only) segment: circle
1.0 1.0   # cartesian coordinates of center
1.0 0.0   # radius and some dummy variable
0.0 6.28... # start and end angles 
0.0 1.0   # parameter interval

1         # number of segments on third boundary
2         # type of third boundary: open nurbs curve
3         # degree n
6         # number of control points m+1
0.0 1.0 0.25 # X,Y,weight of first control point
...
1.0 1.0 0.25 # X,Y,weight of last control point
0.0       # node vector entry 0 to n
0.1       # node vector entry n+1
...       
0.9       # node vector entry m
1.0       # node vector entry m+1 to n+m+1
0.0 1.0   # parameter interval

1         # number of segments on fourth boundary
3         # type of third boundary: closed nurbs curve
3         # degree n
6         # number of control points m+1
0.0 1.0 0.25 # X,Y,weight of first control point
...
1.0 1.0 0.25 # X,Y,weight of last control point
0.0       # node vector entry 0
...
0.1       # node vector entry m+1
0.0 1.0   # parameter interval

1           # number of segments on this boundary
4           # type: analytic
source file # path to source file
\end{verbatim}
\end{code}





%%%%%%%%%%%%%%%%%%%%%%%%%%%%%%%%%%%%%%%%%%%%%%%%%%%%%%%%%%%%%%%%%%%%%%%%%%%%%%%5

\subsection*{Mesh file}

Each mesh file starts with the following header:

\begin{code}{FEAST 2D -- mesh}{Spezifikation_feast_2d_numbers2}
\begin{verbatim}
FEAST_MESH         # ID-tag
2D                 # mode
3                  # version major
0                  # version minor
0                  # type of mesh (0=quad,1=tri,2=hybrid)
25                 # number of macro nodes
16                 # number of macro elements
40                 # number of macro edges
\end{verbatim}
\end{code}

\paragraph{Nodes} FEAST2D supports three kind of nodes: inner nodes (cartesian nodes), 
parameterized nodes and fictious boundary nodes. Parameterisation is equal to FEAT2D, fictitious 
boundary nodes are used in moving boundary scenarios.

\begin{code}{FEAST 2D -- Nodes}{Spezifikation_feast_2d_nodes}
\begin{verbatim}
# Nodes
0.0  0.0 0  # X,Y,0 for inner node
0.0 0.0 -5  # X,Y,nodal property for inner nodes carrying an 
            # additional nodal property
0.25 0.0 1  # parameter value 0.25, dummy, boundary number 1
# ...
\end{verbatim}
\end{code}
As the boundary list is indexed from 1 upwards, inner nodes can be distinguished from boundary 
nodes unambiguously by testing the third value of each line against zero. 
\paragraph{Edges}

Edges are defined by giving the numbers of start and end nodes.

\begin{code}{FEAST 2D -- Edges}{Spezifikation_feast_2d_edges}
\begin{verbatim}
# Edges
1 2  # index of start and end node
# ...
\end{verbatim}
\end{code}

\paragraph{Elements}
Elements are defined using tons of information: First up the numbers of the nodes that make up the macro, 
listed counterclockwisely for both quads and triangles. Next the numbers of the edges, from first node to 
second node, from second to third and so on. The next couple of lines contain neighbourhood information: 
First up the numbers of those elements sharing a common edge, then the numbers of those elements sharing a 
common node but not an edge. Number 0 is used to mark that there is no neighbour. There can be at most one 
edge neighbour per edge, but an arbitrary number of node neighbours. The following trick is used: 
instead of printing out the number of the neighbour element, the count of neighbours is printed out as 
a negative integer. For all such negative tags an additional line is inserted directly after the node 
neighbour line containing the corresponding element neighbours. The third part of the macro definition 
lists anisotropic refinement data. This is coded into five numerical values:
\begin{enumerate}
\item Initial refinement level both in $x$ and in $y$ direction
\item Refinement control is coded into four bits. The LSB controls refinement in $x$ direction, the LSB+1 
controls direction (left or right). The MSB and MSB-1 contain this information for the $y$ direction. 
The integer representing this information is stored.
\item The last three double values give the refinement factors the underlying algorithm uses.
\end{enumerate}
\begin{code}{FEAST 2D -- Elements}{Spezifikation_feast_2d_elements}
\begin{verbatim}
# Macro number 1
0        # type (0=quad, 1=tri)
1 2 7 6  # indices of nodes
1 2 3 4  # indices of edges
0 2 5 0  # Indices of edge neighbours
0 6 0 0  # Indices of node neighbours
1 0 0.5 0.5 0.5   # refinement level, refinement mode
                  # three factors
# cell number 2
1        # type (0=quad, 1=tri)
1 2 7    # indices of nodes
1 2 3    # indices of edges
0 2 5    # Indices of edge neighbours
0 6 0    # Indices of node neighbours
1 2      # refinement level and subdivision mode
# ...
\end{verbatim}
\end{code}


%%%%%%%%%%%%%%%%%%%%%%%%%%%%%%%%%%%%%%%%%%%%%%%%%%%%%%%%%%%%%%%%%%%%%%%%%%5

\subsection*{Partition file}
For each element, the partition file contains one single line containing the
indices of the matrix and parallel blocks for each element. Element numbering
is based on the \texttt{fmesh} file.
\begin{code}{FEAST 2D -- partition}{Spezifikation_feast_2d_numbers3}
\begin{verbatim}
FEAST_PART         # ID-tag
2D                 # mode
3                  # version major
0                  # version minor
47                 # number of elements
4                  # number of parallel blocks
1 4                # parallel and matrix block for element 1
...
\end{verbatim}
\end{code}




%%%%%%%%%%%%%%%%%%%%%%%%%%%%%%%%%%%%%%%%%%%%%%%%5

\subsection*{Boundary condition file}

This file defines the boundary conditions. The user can extend the default boundary conditions
dirichlet and neumann with self defined condition tags (user defined boundary status). 
Also he can add additional information to each boundary specification (user defined boundary information).
Several boundary condition building blocks can be defined. Each block has as default zero dirichlet condition.
Each line in such a block defines a certain condition on a given interval or point. The interval can be
defined via parameter value (\verb+PDEFINT+ with \verb+PAR1 PAR2+) or via node
indices (\verb+NDEFINT+ with \verb+NODE1 NODE2+) (note that parameter
intervals are only unique for a single boundary, thus, boundary references are
given if parameters are used), 
a point can also be defined via parameter value (\verb+PDEFNODE+ with \verb+PAR1+) or via node index (\verb+NDEFNODE+ with \verb+NODE1+).
The boundary value can be a float constant or a tag, telling the library to call a user defined function.

\begin{code}{FEAST 2D -- boundary conditions}{Spezifikation_feast_2d_bc}
\begin{verbatim}
# FEAST2D boundary condition file
#
FEAST_BC           # ID-tag
2D                 # mode
3                  # version major
0                  # version minor
#
2 # number of user defined boundary status (UDBS)
SUPERSONICIN
SUPERSONICOUT
#
2     # number of user defined boundary information
# name INT|FLOAT
USERVALUE1 INT
USERVALUE2 FLOAT
#
3     # number of bcbb (boundary condition building blocks)
#
# bcbb 1
U1    # name (string constant)
3     # number of bcbbd (bcbb definitions)
PDEFINT DIR 0.0 0.0 1.0 1 2.0   
# bcbbd :  PDEFINT|PDEFNODE|NDEFINT|NDEFNODE 
#          DIR|NEU|UDBS1|..|UDBSn 
#          BVALUE|FUNC 
#          BDIX PAR1 [PAR2] | NODE1 [NODE2]
#          USERVALUE1|FUNC..USERVALUEn|FUNC
#
#    if FUNC is given, for parameter evaluation a user 
#    given routine 
#    pboundaryvalue(bs,bcbb,bcbbd,par1,par2,par,valueindex) 
#    is called
#
PDEFNODE NEU FUNC 2.0 1 2.0 
NDEFINT SUPERSONICIN 0.0 1 2 1 2.0
#
# bcbb 2
U2
1
PDEFINT DIR 0.0 0.0 1.0
#
# bcbb 3
P
2
NDEFINT NEU 0.0 2 4
NDEFNODE DIR 10.0 3
\end{verbatim}
\end{code}


%#######################################################################################

\section{Data file}

Extension: {\bf .dat}

At the moment the entries in this file must have the same sequence and
arrangement. The entries are the following

\verb+PROJECTID:+ deprecated

\verb+PROJECTDIR:+ This defines the complete absolute path without \verb+~+
 to the feast subfolder

\verb+LOGDIR:+ This defines the complete absolute path without \verb+~+ to the folder, where
the log files should be saved. This could be a tmp file area.

\verb+MSG_LEVEL:+     level of verbosity  ( 0 = none, 9 = all messages )

\verb+GRIDFILE:+ This is a path relatively to PROJECTDIR to the grid file describing
the domain

\verb+LOADBALANCE:+ deprecated, should be 0

\verb+LOADBALFILE:+ deprecated

\verb+AUTOPARTITION:+  YES/NO This switch defines, if the auto partition is used.
If it is activated, then the number of started processes is used to get the number
of partitions, otherwise the information in the FEAST file is used.

\verb+WEIGHTEDPART:+  YES/NO This switch defines if in case of auto partititon 
load balancing is used.

\verb+CONSTMALMUL:+  YES/NO This switch determines if the faster constant matrix
multiplication is used, if possible.

\verb+FASTMATASS:+ deprecated

\verb+CALCSMOOTHNORM:+  YES/NO This switch defines if after a smoothing operation
in the multigrid driver the norm of the defect is calculated. This is not
necessary for the algorithm, only for debugging purposes. For production runs
you should set this to NO.

\verb+LONGFORM:+ This defines if the output of the master program is in long or
short format.

The next area describes the discretisation of the problem. A discretisation
consists of a finite element and a cubature formula. For efficency reasons
you have the possibilty to define more than one bilinear forms ('matrix')
over one discretisation. 

The entry NDISCR defines the number of discretisation. For each
discretisation, the entries \verb+ELE+, \verb+NBF+ and \verb+BF+ have to be defined. 
The \verb+ELE+
identifier defines the type of the element used for test and ansatz functions,
and the used quadrature formulas for building the matrix, the right hand side and for 
incorporating Neuann boundary conditions. Therefore, the last cubature formula maut be a
1D cubature formula.
Currently only the quadraliteral bilinear Finite Element with the id \verb+Q1+ is fully
supported. For the quadrature formulas have a look to the description of the
cubature module. The entry \verb+NBF+ determines the number of bilinear forms defined
in this discretisation. The last entry \verb+BF+ describes the bilinear form
itself. As example let us have a look to the discretisation of the 
Poisson equation

$$
        -\Delta u = f 
$$
with the weak formulation
$$
c_1 (a_x,b_x) + c_2 ( a_y, b_y ) = (f,b). 
$$

The first integer of the \verb+BLF+ entry describes the number of terms in the bilinear form,
here two.

Every following line in the BLF entry describes one term as follows: coefficient,
ansatz function, test function. If the coefficent is set to \verb+VAR+, then the
function \verb+pcoeff+ from the \verb+userdef+ module is used to calculat the
value of the coefficient. For the ansatz and test functions $\varphi$ the following values
can be used:

\verb+FUNC+  -- $\varphi$\\
\verb+DERX+  -- $\varphi_x$\\
\verb+DERY+  -- $\varphi_y$\\
\verb+DERXX+ -- $\varphi_{xx}$\\
\verb+DERXY+ -- $\varphi_{xy}$\\
\verb+DERYY+ -- $\varphi_{yy}$\\

The next parameter defines the number of terms of the right hand side, in
this case one term, the function itself. 

The last parameter is an string identifier for the matrix in short and long
form.

The next section describes the assignment of the \scarc{} algorithms to the
matrices. The entry  NSCARC defines the number of \scarc{} files to be
included. The next entries consists of two parameters each line, first
parameter the relative path to the \scarc{} algorithm file (format described
later) and the identifier of the matrix, the algorithm should solve. This
assignment can be changed later in the code by API routines. See the
description of the sample program.

\verb+OUTPUTFILE:+    name of the gmv/avs file without suffix\\
\verb+OUTPUTLEVEL:+   the multigrid level, on which the solution shall be
                      written in the output file\\
\verb+LASTSOLUTION:+  deprecated\\
\verb+LASTMATRIX:+    deprecated\\
\verb+STORAGE_DES:+   number of storage descriptors, problem size dependent
\verb+STORAGE_SIZE:+  heap size\\
\verb+MAXMGLEVEL:+    maximum multigrid level\\
\verb+ANISOREFMODE:+  deprecated, should be MODE2

The following entries allows the setting of a certain SBBLAS implementation.

\verb+SBBLASMV:+  implementation of the MV algorithm\\
\verb+SBBLASPRSLOWL:+  implementation of the TriGS algorithm\\
\verb+SBBLASPRSLINE:+  implementation of the MVline algorithm\\
\verb+SBBLASPRSLINE:+  implementation of the Tri algorithm\\
\verb+SBBLASOVERRIDE:+  is it is set to YES, the above defined values are used\\
\verb+SBBLASCONF:+  path to the sbblas.conf file



The further entries in this file are application dependent.

%#######################################################################################

\section{\scarc{} application file}

extension {\bf .scarc}

This file describes the structure of a \scarc{} algorithm over the
various hierachical layers.

The definition starts with the keyword SOLVER.

At the moment, two main solvers are supported, multigrid (MG) and 
conjugate gradient (CG/BICG) schemes. When the latter are intendend to work as
smoothers in an outer multigrid scheme (e.g. when solving multidimensional problems)
then this can be indicated by CG\_SMOOTH and BICG\_SMOOTH, respectively. (The effect is
that the necessary structures are allocated on all multgrid levels). The direct solvers are intented for solving the coarse grid problem in the multigrid scheme.

\begin{verbatim}
SOLVER=[CG|CG_SMOOTH],[iter],[exit-crit],[Hlayer]
  PREC=ALL,0,[omega],[scarcflag],[Hlayerflag]
  [SOLVER|LSMOOTHER]

SOLVER=[BICG|BICG_SMOOTH],[iter],[exit-crit],[Hlayer]
  PREC=ALL,0,[omega],[scarcflag],[Hlayerflag]
  [SOLVER|LSMOOTHER]  

SOLVER=MG,[iter],[exit-crit],[cycle],[npresmooth],[npostsmooth],[Hlayer]
  SMOOTH=ALL,0,[omega],[scarcflag],[Hlayerflag]
  [SOLVER|LSMOOTHER]
  COARSE=[SOLVER]

SOLVER=GLOBAL

SOLVER=PGLOBAL

SOLVER=MGLOBAL
 
LSMOOTHER =  JACOBI|TRIGS|ADITRIGS|ILU
\end{verbatim}

\begin{itemize}
\item \verb+[iter]+  : iteration count
\item \verb+[exit-crit]+ : exit criterion absolute (\verb+ABS+) or relative (\verb+REL+). \verb+ABS:1E-6+
	    e.g. indicates, that the iteration process shall end, if the absolute residual is smaller than 
	    $10^{-6}$.  
\item \verb+[omega]+ : damping parameter for this level
\item \verb+[Hlayer]+ : hierachical layer this scheme works on
\item \verb+[scarcflag]+ : If this parameter is 0, the next scheme is a simple local
              smoother LSMOOTHER, otherwise a further \scarc{} solver.
\item \verb+[Hlayerflag]+ : if this parameter is 0, the next scheme works on a different
               hierachical layer, otherwise on the same.
\item \verb+[cycle]+ : multigrid cycle type V,W or F
\item \verb+[npresmooth],[npostsmooth]+ : number of pre- and postsmoothing steps
\end{itemize}

Lets have a look to the following example:

\begin{tabbing}
\verb+SOLVER=CG,32,REL:1.0E-6,HL_SD+ \phantom{xxxxxxx} \=   The global scheme is a CG scheme which stops after \\
				      \> 32 iterations or if a relative accuracy of $10^{-6}$ is reached. \\
\\
\verb+  PREC=ALL,1,1.0,HL_SD,1,1+   \>   The preconditioning is done with damping factor 1.0.  \\
                                    \>   The precondiotioner is of \scarc{} type and acts on \\
				    \>   the same hierachical level.  \\
\\    
\verb+SOLVER=MG,1,REL:1.0E-6,V,1,1,HL_SD+  \>  As preconditioner acts a global multigrid scheme  \\
\verb+SMOOTHER=ALL,0,0.8,HL_SD,1,0+  \>   which performs one global step. The exit criterion  \\
                                     \>    has in this case no meaning. The global multigrid  \\
				     \>   performs 1 pre- and 1 postsmoothing step. The  \\
				     \>   preconditioner is of \scarc{} type and works not on \\
				     \>   the same hierachical level.  \\
\\				   
\verb+  SOLVER=MG,1,REL:1.0E-6,F,4,4,HL_PB+   \> As smoother works a local multigrid scheme,   \\
\verb+    SMOOTHER=ALL,0,1.0,HL_PB,0,0+   \>   which works seperatly on the parallel subdomains  \\
                                           \> It performs one step and 4 pre- and 4 postsmoothing \\
					\>  steps. The smoother for this scheme is a local smoother. \\
					\> In this case the last parameter has no meaning.  \\
\\      
\verb+      LSMOOTHER=TRIGS+      \>            Local smoother triGaussSeidel  \\
\\	
\verb+    COARSE=CG,512,REL:1.0E-6,HL_PB+  \>   The coarse grid solver for the local multigrid scheme \\
\verb+      PREC=ALL,1,1.0,HL_PB,0,0+   \>     It performs up to 512 steps till an accuracy of relatively  \\
\verb+        LSMOOTHER=JACOBI+      \>        $10^{-6}$ is reached. It is preconditioned with the Jacobi method.\\
\\	  
\verb+  COARSE=CG,512,REL:1.0E-6,HL_SD+    \>        The coarse grid solver for the global multigrid scheme.\\
\verb+    PREC=ALL,1,1.0,HL_SD,0,0+     \>       It has the same parameters as the local scheme.\\
\verb+      LSMOOTHER=JACOBI+
\end{tabbing}

