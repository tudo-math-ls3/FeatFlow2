\section{Documentation}

To keep the documentation up-to-date, the documentation for the
library functions and data structures is direct included in the source
code. This information is structered by special tags which are comments
for the fortran compiler. The documentation can be produced in two
formats, HTML and TeX. To the TeX output there are some chapters added
which contains basic information concerning theoretical background, data
formats, tutorials, benchmark etc. To generate the HTML-documentation,
run the script

\verb+~/feast/feast/docs/html/generate_html+

to generate the TeX-documentation run the script

\verb+~/feast/feast/docs/tex/generate_tex+


\subsection{Header}

The header information is tagged by the \verb+<name>+ tag, which defines
the name of the module, and the \verb+<purpose>+ tag, which describes the
global purpose of the module.

\begin{code}{Header documentation}{header_doc}
\begin{verbatim}
########################################################################
#                                                                      #
#                   FINITE ELEMENT ANALYSIS TOOLS 2                    #
#                                                                      #
# Authors: M. Koester, M. Moeller, S. Turek, S. Buijssen               #
#                                                                      #
#                                                                      #
# Contact: Applied Mathematics, TU Dortmund University                 #
#          Vogelpothsweg 87, 44227 Dortmund                            #
#          Germany                                                     #
#                                                                      #
# Web:     http://www.featflow.de/en/software/featflow2.html           #
#          mailto:featflow@featflow.de                                 #
#                                                                      #
!########################################################################
!#                                                                      #
!# <name> fsystem </name>                                               #
!#                                                                      #
!#                                                                      #
!# <purpose>                                                            #
!# System routines like time, string/value conversions etc.             #
!# </purpose>                                                           #
!#                                                                      #
!########################################################################
\end{verbatim}
\end{code}

\subsection{Global constants}


Global constants are documented by the tags \verb+<constants>+ and 
\verb+<constantblock>+, this tag supports the \verb+description+ attribut
for further explanation. The comments for every constant definition must
be placed above the corresponding declaration.

\begin{code}{Global constants documentation}{gconst_doc}
\begin{verbatim}
!<constants>

!<constantblock description="constants for logical values">

  ! logical value 'true'
  integer, parameter :: YES = 0

  ! logical value 'false'
  integer, parameter :: NO = 1

!</constantblock>


!<constantblock description="kind values">

  ! kind value for double precision
  integer, parameter :: DP = selected_real_kind(13,307)

  ! kind value for single precision
  integer, parameter :: SP = selected_real_kind(6,63)

  ! kind value for 32Bit integer
  integer, parameter :: I32 = selected_int_kind(8)

  ! kind value for 64Bit integer
  integer, parameter :: I64 = selected_int_kind(10)

!</constantblock>
!</constants>
\end{verbatim}
\end{code}

\subsection{Global types}

Global types are documented by the tags \verb+<types>+ and 
\verb+<typeblock>+, this tag supports the \verb+description+ attribut
for further explanation. The comments for every definition must
be placed above the corresponding declaration.

\begin{code}{Global types documentation}{gtype_doc}
\begin{verbatim}
!<types>

!<typeblock>

  ! iteration descriptor
  type t_iterator

    ! start value
    integer :: istart

    ! stop value
    integer :: istop

    ! step value
    integer :: istep

  end type
!</typeblock>


!<typeblock>

  ! simulation of an array of double pointers
  type t_realPointer
    real(DP), dimension(:), pointer :: ptr
  end type t_realPointer

!</typeblock>

!</types>
\end{verbatim}
\end{code}

\subsection{Global variables}

Global variables are documented by the tag \verb+<globals>+.
The comments for every definition must
be placed above the corresponding declaration.

\begin{code}{Global variables documentation}{gvar_doc}
\begin{verbatim}
!<globals>

  ! global system configuration
  type (t_sysconfig) :: sys_sysconfig

  ! parallel mode 0 = single, 1=double binary (deprecated)
  integer :: sys_parmode

  ! output format
  integer :: coutputFormat

  ! output format
  character(len=256) :: soutputFileName

!</globals>

\end{verbatim}
\end{code}

\subsection{Functions}

The documentation for a function is tagged by the \verb+<function>+ tags, which
must include the function definition. The \verb+<function>+ tag can contain the following
tags for additional documentation:

\begin{itemize}
\item \verb+<description>+: contains description of the function
\item \verb+<result>+: contains description of the result of the function
\item \verb+<input>+: contains description of the input parameters of the function
\end{itemize}

\begin{code}{Function documentation}{func_doc}
\begin{verbatim}
!<function>
  character (len=32) function sys_sd(dvalue, idigits) result(soutput)

    !<description>
    ! This routine converts a double value to a string with idigits
    ! decimal places.
    !</description>

    !<result>
    ! String representation of the value, filled with white spaces.
    ! At most 32 characters supported.
    !</result>

    !<input>

      ! value to be converted
      real(DP), intent(in) :: dvalue
    
      ! cont of digits
      integer
                    :: idigits
    !</input>
!</function>
\end{verbatim}
\end{code}


\subsection{Subroutines}

The documentation for a subroutine is tagged by the \verb+<subroutine>+ tags, which
must include the subroutine definition. The \verb+<subroutine>+ tag 
can contain the following tags for additional documentation:

\begin{itemize}
\item \verb+<description>+: contains description of the function
\item \verb+<result>+: contains description of the result of the function
\item \verb+<input>+: contains description of the input parameters of the function
\item \verb+<output>+: contains description of the output parameters of the function
\item \verb+<inputoutput>+: contains description of the input/output parameters of the function
\end{itemize}

\begin{code}{Subroutine documentation}{sr_doc}
\begin{verbatim}
!<subroutine>
    subroutine sys_parInElement(Dcoord, dxpar, dypar, dxreal, dyreal)

    !<description>
    !This subroutine is to find the parameter values for a given point (x,y) in real
    !coordinates.
    !
    !Remark: This is a difficult task, as usually in FEM codes the parameter 
    !values are known and one wants to obtain the real coordinates.
    !</description>

    !<input>

      !coordinates of the evaluation point
      real(DP),intent(in) :: dxreal,dyreal

      !coordinates of the element vertices
      real(DP), dimension(2,4),intent(in) :: Dcoord

    !</input>

    !<output>

      !parameter values of (x,y)
      real(DP), intent(out) :: dxpar,dypar

    !</output>

!</subroutine>

\end{verbatim}
\end{code}

