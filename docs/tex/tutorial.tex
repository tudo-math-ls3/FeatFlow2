\chapter{Tutorial}

\section{A first step to FEAST Finite Element \& Solution Tools}

In the first step we want to create our grid. For this task we use
the \devisor program, which you have to load and install separately
from \url{http://www.featflow.de}.

Start \devisor with the command

\begin{verbatim}
grid3
\end{verbatim}

First we build a new domain. Go to

\verb+Domain->New Domain+ 

type \verb+1,0+ for every size in the dialog.


then we want to build the boundary description. Press the
space bar and select 

\verb+Mouse mode->Create SegmentList+, 

then
go to position (0,0) and klick the
left button, go to (1,0),(1,1),(0,1) and
klick again. To finish the boundary description press the
space bar and select 

\verb+Done!+. 

Then go to 

\verb+Domain->Save Domain As->FEAST V3+

and save the 
file under
the name \verb+~/feast/feast/grids/firstgrid.feast+.

In the next step we will create the nodes. First, we define the nodes
on the boundary. This will be done automatically by the the function

\verb+Coarse Mesh->Add multiple parameter nodes(2D)+. 

Select this function and a dialog will appear, where you can input some parameters. Change the value
of the \verb+Number of Nodes+ from 4 to 8 and click OK. This will create boundary nodes
on the edge and midpoints. One interior point has to be defined, to do this, press
space bar and select 

\verb+Mouse mode->Create Points+

and klick with the left button to
the position (0.5,0.5).

In the next step you have to define the edges. Press space bar and select 

\verb+Select+.


Klick with the left button and draw
the rubberband with pressed button so, that the nodes 1 and 2 are gathered. Then
press the E key. A new edge is created. Repeat this for the node groups
(2,9), (9,8), (8,1), (9,6), (6,7), (7,8), (5,6), (4,2), (9,4), (3,4) and (2,3)).
Then gather the complete domain, press space bar and select 

\verb+Add Quads+.

Now the grid is ready and you can finally save it.

Now we can try to start a calculation run. Go to the folder 

\verb+~/feast/feast/applications/poisson+ 

and type: \verb+./configure+

This creates the makefile.

Type \verb+make+ to build the executable.

After editing and building we have to modify the central data file
called \verb+master.dat+. In this file there are several parameters concerning
the environment and simulation defined. The detailed format of this file
is described later. Make sure that the lines with PROJECTDIR and LOGDIR
contains the correct path to your FEAST folder.
 
Before you start the executable make sure that the LAM daemon is
running. To achieve this type the following commands

\begin{verbatim}
wipe
lamboot
\end{verbatim}

to start the program type

\begin{verbatim}
mpirun -c 4 doenerkebap
\end{verbatim}


The program run should start with 

\begin{verbatim}
* master: This is FEAST!  Version 0.9.0 (22.11.2005 RC1)
\end{verbatim}

and end with 

\begin{verbatim}
* master: Get finish message from    1 with                0
* master: Get finish message from    2 with                0
* master: Get finish message from    3 with                0
\end{verbatim}


Now have a look to the solution files \verb+sol_simple.gmv+. To
view this file you will need the program {\em GMV}.


\section{Sample program}

The following sample program demonstrates the solving of the poisson problem
with the FEAST library. This sample program is the slavemod\_complex.f90 file
in the \verb+~/feast/feast/applications/poisson+ folder.

\verbatiminput{../../applications/poisson/slavemod_complex.f90}


