
Please perform the following prelimarary installation steps:

get \verb+feast_xx.xx.xx.tar.gz+ from \verb+www.featflow.de+

Untar the file with the command

\begin{verbatim}
gzip -d feast_xx.xx.xx.tar.gz
tar xvf feast_xx.xx.xx.tar
\end{verbatim}

This creates the following directory structure

\begin{tabbing}
feast \phantom{xxxxxxxx} \= feast \phantom{xxxxxxxxxxxxxxx} \=  -FEAST program- \\
      \> doc.pdf  \> -This document-\\
      \> README  \> -latest information-\\
      \> CHANGES \> -changes from previous versions-\\
      \> OPENBUGS \> -known and open bugs-\\
\end{tabbing}

The feast folder itself consists of the following subfolders

\begin{tabbing}
\verb+applications+ \phantom{xxxxxxxx} \= application programs \\
\verb+arch+ \> architecture dependent stuff\\
\verb+bin+ \> shell scripts and binaries\\
\verb+docs+ \> Documentation\\
\verb+fbenchmark2+  \> benchmark program \\
\verb+grids+ \> domain descriptions \\
\verb+kernel+ \> FEAST kernel sources\\
\verb+libraries+ \> external libraries \\
\verb+object+ \> folder for object files \\
\verb+outofdata+ \> stuff out of date\\
\verb+sbblas+ \> SBBLAS folder \\
\verb+sbblasbench+ \> SBBLAS benchmark folder \\
\verb+scarc+   \> folder containing the scarc algorithm descriptions\\
\verb+solver+   \> folder containing the solver algorithm descriptions
\end{tabbing}

      
Get LAM-MPI, a message passing library, from

\verb+http://www.lam-mpi.org/download+

and install it in accordance with the instructions found
in the \verb+LAM+ folder (for lsiii users: already installed) \cite{BurnsDaoudVaigl1994,SquyresLumsdaine2003}

Make sure that the environment variables \verb+MPI_INC+ and 
\verb+MPI_LIB+ point to the directories for include files and the LAM/MPI libraries, e.g.

\begin{verbatim}
setenv MPI_INC /usr/local/lam/include
setenv MPI_LIB /usr/local/lam/lib
\end{verbatim}

Make sure, that the path \verb+/usr/local/lam/bin+ is in your 
\verb+PATH+ variable.


FEAST has a global configure script located in the \verb+bin+ folder. Every
application subfolder contains a local configure script which calls the
global script with application specific options. 

\verb+configure+ tries to detect the system architecture. The architecture is
coded as following:

\verb+arch-cpu-os-mpi-compiler-blas+

The sub specifications means as follows:
\begin{itemize}
\item \verb+arch+: architecture, currently \verb+alpha,ibm,pc,sun4u,sx6+
\item \verb+cpu+: processor type, currently \verb+athlon,athlon64,athlonxp,ev6,opteron,pentium3,pentium4,+\\
\verb+pentium4m,powerpc_power4,sparcv8,sparcv9+
\item \verb+os+: operating system, currently \verb+aix,linux32,linux64,osf1,sunos,superux+
\item \verb+mpi+: MPI environment, currently \verb+dmpi,lammpi,mpich,optmpich,tsmpi+
\item \verb+compiler+: compiler, currently \verb+cf90,ifort,g95,gfortran,pgi,sunf90,xlf+
\item \verb+blas+: used BLAS library, currently \verb+atlas,blas,dxml,essl,goto,perf+
\end{itemize}


Currently, the following IDs are supported:
\verbatiminput{valid_ids}

The configure script supports the following options:
\verbatiminput{configure_help}

After configure is successfully run a makefile is created. Type \verb+make+
to compile the package and the application.
