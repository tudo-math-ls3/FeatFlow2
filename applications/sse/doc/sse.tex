\documentclass[11pt,a4paper]{article}
\usepackage[utf8]{inputenc}
\usepackage[english]{babel}
\usepackage{amsmath}
\usepackage{amsfonts}
\usepackage{amssymb}
\usepackage[left=2cm,right=2cm,top=2cm,bottom=2cm]{geometry}
\usepackage{hyperref}
\title{Three-Dimensional Idealized Model for Tidal Motion in Tidal Estuaries. Part II: An Efficient Finite-Element Multigrid Approach}
\author{Matthias M\"oller}
\begin{document}
\maketitle

\begin{abstract}
...
\end{abstract}

\section{Introduction}
...

\section{Mathematical Model}

The linear elliptic partial differential equation for the sea surface elevation as derived in [?] reads
\begin{alignat}{2}
\nabla\cdot(\mathbb{A}\nabla N)+i\omega N&=0  \quad && \text{in } \Omega,
\label{eq:sse_scalar1}\\
N&=A_{M_2} \quad && \text{on } \Gamma_D,
\label{eq:sse_scalar2}\\
(\mathbb{A}\nabla N)\cdot\mathbf{n}&=0 \quad && \text{on } \Gamma_N,
\label{eq:sse_scalar3}
\end{alignat}
where $\Gamma_D$ and $\Gamma_N$ denote the Dirichlet and Neumann boundary part and $A_{M_2}$ is the amplitude of sea surface elevation at the Dirichlet boundary. Both the sea surface elevation $N\in\mathbb{C}$ and the matrix
$$
\mathbb{A}=
\begin{bmatrix}
C_1 & C_2\\
C_3 & C_4
\end{bmatrix},\quad\text{where}\quad C_k\in\mathbb{C}, k=1,\dots,4
$$
consist of complex values, where $i$ denotes the imaginary number, i.e. $i^2=-1$.

Let us define the $z$-dependent function $C:\mathbb{C}\mapsto\mathbb{C}$ as
\begin{equation}
C(\alpha,z)=\frac{g}{\alpha^3A_v}\left[\frac{\alpha^2A_vz\sinh(\alpha h)-s\sinh(\alpha z)+\alpha z s \cosh(\alpha h)}{\alpha A_v\sinh(\alpha h)+s\cosh(\alpha h)}\right]
\end{equation}
and introduce the auxiliary real-valued parameters
$$
\alpha_1=\sqrt{i\frac{\omega+f}{A_\nu}},\qquad
\alpha_2=\sqrt{i\frac{\omega-f}{A_\nu}}.
$$
Then the complex-valued coefficients $C_1,\dots,C_4$ are given by
\begin{alignat}{4}
C_1&=&\frac{C(\alpha_1,-h)+C(\alpha_2,-h)}{2},&& \qquad
C_2&=&i\frac{C(\alpha_1,-h)-C(\alpha_2,-h)}{2},\\
C_3&=&-i\frac{C(\alpha_1,-h)-C(\alpha_2,-h)}{2},&&\qquad
C_4&=&\frac{C(\alpha_1,-h)+C(\alpha_2,-h)}{2}.
\end{alignat}
Note that $C_1=C_4$ and $C_2=-C_3$. The other parameters are given in the following table

\begin{center}
\begin{tabular}{|l|l|l|l|}
\hline
Symbol & Value & Description & Units\\
\hline
$(x,y,z)$ & & Cartesian coordinates & $m$\\
$t$ & & time & $s$\\
$L$ & & length scale & $m$\\
$H$ & & mean depth & $m$\\
$h(x,y)$ & 10 & bed profile & $m$\\
$\eta(x,y,t)$ & & sea surface elevation above $z=0$ & $m$\\
$\mathbf{u}(x,y,z,t)$ & & velocity vector & $m s^{-1}$\\
$f(x,y)$ & 0 & Coriolis acceleration (Earth rotation) & $s^{-1}$\\
$g$ & 10 & gravitational acceleration & $ms^{-2}$\\
$A_h(x,y)$ & & horizontal eddy viscosity & $m^2 s^{-1}$\\
$A_\nu(x,y)$ & 0.012 & vertical eddy viscosity & $m^2 s^{-1}$\\
$s(x,y)$ & 0.049 & bottom stress & $m s^{-1}$\\
$A_{M_2}(x,y)$ & 1.0 & $M_2$ amplitude at seaward side & $m$\\
$\omega$ & $1.4\times 10^{-4}$ &tidal frequency of $M_2$ tide & $s^{-1}$\\
\hline
\end{tabular}
\end{center}

\subsection{Variational formulation}
The variational formulation reads: find $N\in V_{A_{M_2}}$ such that
\begin{equation}
\int_\Omega\nabla \varphi\cdot(-\mathbb{A}\nabla N)+ i\omega \varphi N\,\mathrm{d}\mathbf{x}=0\qquad \forall \varphi\in V_0,
\end{equation}
The essential boundary conditions have been built into the trial and test spaces
\begin{align}
V_{A_{M_2}}&=\{v\in H^1(\Omega)\,:\, v=A_{M_2}\text{ on }\Gamma_D\},\\
V_0&=\{\varphi\in H^1(\Omega)\,:\, \varphi=0\text{ on }\Gamma_D\}.
\end{align}

\subsection{Finite element approximation}
Let the variable for the sea surface elevation be approximated by finite elements as follows
\begin{equation}
N(\mathbf{x})=\sum_{j}\varphi_j(\mathbf{x})N_j,
\end{equation}
where $\{\varphi_j(\mathbf{x})\}_j$ denotes a basis of \emph{real} valued functions and complex valued coefficients $N_j\in\mathbb{C}$.

Instead of working with complex values directly we split them into their real and imaginary parts, i.e. $N_j=N_j^\mathrm{Re}+iN_j^\mathrm{Im}$ and use separate scalar finite element fields $N^\mathrm{Re}$ and $N^\mathrm{Im}$ for discretizing each one of them. Here and below superscript $(\cdot)^\mathrm{Re}$ and $(\cdot)^\mathrm{Im}$ denote the real and imaginary part, respectively. Then the problem at hand reads: find $(N^\mathrm{Re},N^\mathrm{Im})\in V_{A_{M_2}^\mathrm{Re}}\times V_{A_{M_2}^\mathrm{Im}}$ such that
\begin{equation}
\sum_j\int_\Omega \nabla\varphi_i\cdot\left(-\mathbb{A}^\mathrm{Re}\nabla N_j^\mathrm{Re}+\mathbb{A}^\mathrm{Im}\nabla N_j^\mathrm{Im}\right)-\omega \varphi_i N_j^\mathrm{Im}\,\mathrm{d}\mathbf{x}=0
\end{equation}
and
\begin{equation}
\sum_ji\int_\Omega \nabla\varphi_i\cdot\left(-\mathbb{A}^\mathrm{Im}\nabla N_j^\mathrm{Re}-\mathbb{A}^\mathrm{Re}\nabla N_j^\mathrm{Im}\right)+\omega \varphi_i N_j^\mathrm{Re}\,\mathrm{d}\mathbf{x}=0
\end{equation}
for all admissible real valued test functions $\varphi_i\in V_{0}$.
To simplify the notation let us define the following auxiliary stiffness matrices and the standard consistent mass matrix
\begin{alignat}{2}
S^\mathrm{Re}&=\{s_{ij}^\mathrm{Re}\} \qquad s_{ij}^\mathrm{Re}&=&\int_\Omega \nabla\varphi_i\cdot(-\mathbb{A}^\mathrm{Re}\nabla\varphi_j)\,\mathrm{d}\mathbf{x},\\
S^\mathrm{Im}&=\{s_{ij}^\mathrm{Im}\} \qquad s_{ij}^\mathrm{Im}&=&\int_\Omega \nabla\varphi_i\cdot(-\mathbb{A}^\mathrm{Im}\nabla\varphi_j)\,\mathrm{d}\mathbf{x},\\
M&=\{m_{ij}\} \qquad m_{ij}&=&\int_\Omega \varphi_i\varphi_j\,\mathrm{d}\mathbf{x}.
\end{alignat}
Then the problem at hand can be written in compact matrix notation as follows:
\begin{equation}
\begin{bmatrix}
S^\mathrm{Re} & -S^\mathrm{Im}-\omega M\\
S^\mathrm{Im}+\omega M &  S^\mathrm{Re}
\end{bmatrix}
\begin{bmatrix}
N^\mathrm{Re}\\
N^\mathrm{Im}
\end{bmatrix}
=
\begin{bmatrix}
0\\
0
\end{bmatrix}.
\end{equation}

\section{First-order system formulation}

In what follows, we present two alternative reformulations of the second-order scalar model problem as a first order system. In essence, the two formulations vary in the treatment of the anisotropic diffusion operator $\mathbb{A}$.

\subsection{Model problems}
Let $\boldsymbol{\sigma}=\mathbb{A}\nabla N$ in order to convert the scalar problem equation \eqref{eq:sse_scalar1}--\eqref{eq:sse_scalar3} into the first-order system
\begin{alignat}{2}
\nabla\cdot\boldsymbol{\sigma}+i\omega N&=0  \quad && \text{in } \Omega,
\label{eq:sse_system1}\\
\boldsymbol{\sigma}-\mathbb{A}\nabla N&=0 \quad && \text{in } \Omega,
\label{eq:sse_system2}\\
N&=A_{M_2} \quad && \text{on } \Gamma_D,
\label{eq:sse_system3}\\
\boldsymbol{\sigma}\cdot\mathbf{n}&=0 \quad && \text{on } \Gamma_N.
\label{eq:sse_system4}
\end{alignat}

It is also possible to define $\boldsymbol{\tau}=\nabla N$ as the gradient of the sea surface elevation which yields the alternative first-order system resulting from the scalar problem equation \eqref{eq:sse_scalar1}--\eqref{eq:sse_scalar3}
\begin{alignat}{2}
\nabla\cdot(\mathbb{A}\boldsymbol{\tau})+i\omega N&=0  \quad && \text{in } \Omega,
\label{eq:sse_system5}\\
\boldsymbol{\tau}-\nabla N&=0 \quad && \text{in } \Omega,
\label{eq:sse_system6}\\
N&=A_{M_2} \quad && \text{on } \Gamma_D,
\label{eq:sse_system7}\\
(\mathbb{A}\boldsymbol{\tau})\cdot\mathbf{n}&=0 \quad && \text{on } \Gamma_N.
\label{eq:sse_system8}
\end{alignat}
Both approaches lead to slightly different variational formulations and, consequently, system matrices.

\subsection{Variational formulation I}
The variational formulation associated with \eqref{eq:sse_system1}--\eqref{eq:sse_system4} reads: find $N\in W$ and $\boldsymbol{\sigma}\in\mathbf{Q}_0$ such that 
\begin{alignat}{2}
\int_\Omega\varphi\left(\nabla\cdot\boldsymbol{\sigma}+i\omega N\right)\,\mathrm{d}\mathbf{x}&=0 \qquad && \forall \varphi\in W,
\label{eq:weak1}\\
\int_\Omega\boldsymbol{\psi}\cdot\left(\boldsymbol{\sigma}-\mathbb{A}\nabla N\right)\,\mathrm{d}\mathbf{x}&=0 \qquad && \forall \boldsymbol{\psi}\in \mathbf{Q}_0,
\label{eq:weak2}
\end{alignat}
where the test and trial spaces are defined as follows:
\begin{equation}
W=L^2(\Omega),\quad
\mathbf{Q}_0=\{\mathbf{q}\in H(\text{div};\Omega)\,:\,\mathbf{q}\cdot\mathbf{n}=0 \text{ on }\Gamma_N\}.
\end{equation}
Note that the Neumann boundary condition \eqref{eq:sse_scalar2} of the second-order scalar problem becomes a Dirichlet boundary condition \eqref{eq:sse_system3} in the first-order system formulation, and hence, it has been implemented into the test and trial spaces. Since $\mathbb{A}^*=\overline{\mathbb{A}}^T$ and $\overline{\boldsymbol{\psi}}=\boldsymbol{\psi}$, it is clear that
\begin{equation}
\boldsymbol{\psi}\cdot(\mathbb{A}\nabla N)
=\boldsymbol{\psi}^T(\mathbb{A}\nabla N)
=(\boldsymbol{\psi}^T\mathbb{A})\nabla N
=(\mathbb{A}^*\boldsymbol{\psi})^*\nabla N
=\overline{\left(\overline{\mathbb{A}}^T\boldsymbol{\psi}\right)}^T\nabla N
=(\mathbb{A}^T\boldsymbol{\psi})\cdot\nabla N.
\end{equation}
Integration by parts is applied to \eqref{eq:weak2} to include the boundary condition \eqref{eq:sse_system3} as a natural one
\begin{align*}
\int_\Omega\boldsymbol{\psi}\cdot(\boldsymbol{\sigma}-\mathbb{A}\nabla N)\,\mathrm{d}\mathbf{x}
&=\int_\Omega\boldsymbol{\sigma}\cdot\boldsymbol{\psi}-(\mathbb{A}^T\boldsymbol{\psi})\cdot\nabla N\,\mathrm{d}\mathbf{x}\\
&=\int_\Omega\boldsymbol{\psi}\cdot\boldsymbol{\sigma}+\nabla\cdot(\mathbb{A}^T\boldsymbol{\psi})N\,\mathrm{d}\mathbf{x}-\int_\Gamma N(\mathbb{A}^T\boldsymbol{\psi})\cdot\mathbf{n}\,\mathrm{d}\mathbf{x}\\
&=\int_\Omega\boldsymbol{\psi}\cdot\boldsymbol{\sigma}+\nabla\cdot(\mathbb{A}^T\boldsymbol{\psi})N\,\mathrm{d}\mathbf{x}-\int_{\Gamma_D} A_{M_2}(\mathbb{A}^T\boldsymbol{\psi})\cdot\mathbf{n}\,\mathrm{d}s-\int_{\Gamma_N} N(\mathbb{A}^T\boldsymbol{\psi})\cdot\mathbf{n}\,\mathrm{d}s
\end{align*}
Thus, the variational formulation for the problem at hand reads: find $(N,\boldsymbol{\sigma})\in W\times\mathbf{Q}_0$ such that
\begin{align}
&\int_\Omega\varphi\left(\nabla\cdot\boldsymbol{\sigma}+i\omega N\right)+\boldsymbol{\psi}\cdot\boldsymbol{\sigma}+\nabla\cdot(\mathbb{A}^T\boldsymbol{\psi})N\,\mathrm{d}\mathbf{x}\\
=&\int_{\Gamma_D} A_{M_2}(\mathbb{A}^T\boldsymbol{\psi})\cdot\mathbf{n}\,\mathrm{d}s+\int_{\Gamma_N} N(\mathbb{A}^T\boldsymbol{\psi})\cdot\mathbf{n}\,\mathrm{d}s \qquad \forall (\varphi,\boldsymbol{\psi})\in W\times\mathbf{Q}_0.
\end{align}

\subsection{Variational formulation II}
The variational formulation associated with \eqref{eq:sse_system5}--\eqref{eq:sse_system8} reads: find $(N,\boldsymbol{\sigma})\in W\times\mathbf{Q}_0$ such that 
\begin{alignat}{2}
\int_\Omega\varphi\left(\nabla\cdot(\mathbb{A}\boldsymbol{\tau})+i\omega N\right)\,\mathrm{d}\mathbf{x}&=0 \qquad && \forall \varphi\in W,
\label{eq:weak3}\\
\int_\Omega\boldsymbol{\psi}\cdot\left(\boldsymbol{\tau}-\nabla N\right)\,\mathrm{d}\mathbf{x}&=0 \qquad && \forall \boldsymbol{\psi}\in \mathbf{Q}_0.
\label{eq:weak4}
\end{alignat}
In order to impose the natural boundary conditions \eqref{eq:sse_system7}, we perform integration by parts in \eqref{eq:weak4} 
\begin{equation}
\int_\Omega\boldsymbol{\psi}\cdot\boldsymbol{\tau}+
\nabla\cdot\boldsymbol{\psi}N\,\mathrm{d}\mathbf{x}=
\int_{\Gamma_D}A_{M_2}\boldsymbol{\psi}\cdot\mathbf{n}\,\mathrm{d}s+
\int_{\Gamma_N}N\boldsymbol{\psi}\cdot\mathbf{n}\,\mathrm{d}s.
\end{equation}
Thus, the variational formulation for the problem at hand reads: find $(N,\boldsymbol{\sigma})\in W\times\mathbf{Q}_0$ such that
\begin{align}
&\int_\Omega\varphi\left(\nabla\cdot(\mathbb{A}\boldsymbol{\tau})+i\omega N\right)+\boldsymbol{\psi}\cdot\boldsymbol{\tau}+\nabla\cdot\boldsymbol{\psi}N\,\mathrm{d}\mathbf{x}\\
=& \int_{\Gamma_D}A_{M_2}\boldsymbol{\psi}\cdot\mathbf{n}\,\mathrm{d}s+
\int_{\Gamma_N}N\boldsymbol{\psi}\cdot\mathbf{n}\,\mathrm{d}s
\qquad \forall (\varphi,\boldsymbol{\psi})\in W\times\mathbf{Q}_0.
\end{align}

\subsection{Finite element approximation}
Let us approximate the sea surface elevation and its derivative by finite elements again separating the real and imaginary parts as in the scalar case. That is,
\begin{alignat}{3}
N(\mathbf{x})&=\sum_{j}\varphi_j(\mathbf{x})N_j && \qquad N_j&=N_j^\mathrm{Re}+iN_j^\mathrm{Im}\\
\boldsymbol{\sigma}(\mathbf{x})=(\sigma_1(\mathbf{x}),\sigma_2(\mathbf{x}))^T&=\sum_{j}\psi_j(\mathbf{x})\boldsymbol{\sigma}_j && \qquad \boldsymbol{\sigma}_j&=\boldsymbol{\sigma}_j^\mathrm{Re}+i\boldsymbol{\sigma}_j^\mathrm{Im}\\
\boldsymbol{\tau}(\mathbf{x})=(\tau_1(\mathbf{x}),\tau_2(\mathbf{x}))^T&=\sum_{j}\psi_j(\mathbf{x})\boldsymbol{\tau}_j && \qquad \boldsymbol{\tau}_j&=\boldsymbol{\tau}_j^\mathrm{Re}+i\boldsymbol{\tau}_j^\mathrm{Im}
\end{alignat}
Note that $\{\psi_j\}$ are scalar basis functions which are multiplied by vector-valued coefficients in order to obtain a vector-valued finite element functions. Here, we do not use vector-valued finite elements such as Raviar-Thomas or Brezzi-Douglas-Marini finite elements.

\paragraph{First formulation.} The problem reads: find $(N^\mathrm{Re},N^\mathrm{Im},\boldsymbol{\sigma}^\mathrm{Re},\boldsymbol{\sigma}^\mathrm{Im})\in [W]^2\times [\mathbf{Q}_0]^2$ such that
\begin{align}
&\sum_j\int_\Omega \varphi_i\left(\nabla\cdot\boldsymbol{\sigma}_j^\mathrm{Re}-\omega N^\mathrm{Im}_j\right)+
\boldsymbol{\psi}_i\cdot\boldsymbol{\sigma}^\mathrm{Re}_j+
\nabla\cdot({(\mathbb{A}^\mathrm{Re})}^T\boldsymbol{\psi}_i)N^\mathrm{Re}_j-
\nabla\cdot({(\mathbb{A}^\mathrm{Im})}^T\boldsymbol{\psi}_i)N^\mathrm{Im}_j
\,\mathrm{d}\mathbf{x}\\
=&
\int_{\Gamma_D}\left(A^\mathrm{Re}_{M_2}({(\mathbb{A}^\mathrm{Re})}^T\boldsymbol{\psi}_i)-
A^\mathrm{Im}_{M_2}({(\mathbb{A}^\mathrm{Im})}^T\boldsymbol{\psi}_i)
\right)\cdot\mathbf{n}\,\mathrm{d}s
+
\int_{\Gamma_N}\left(N^\mathrm{Re}_j({(\mathbb{A}^\mathrm{Re})}^T\boldsymbol{\psi}_i)-
N^\mathrm{Im}_j({(\mathbb{A}^\mathrm{Im})}^T\boldsymbol{\psi}_i)
\right)\cdot\mathbf{n}\,\mathrm{d}s
\end{align}
and
\begin{align}
&\sum_ji\int_\Omega \varphi_i\left(\nabla\cdot\boldsymbol{\sigma}_j^\mathrm{Im}+\omega N^\mathrm{Re}_j\right)+
\boldsymbol{\psi}_i\cdot\boldsymbol{\sigma}^\mathrm{Im}_j+
\nabla\cdot({(\mathbb{A}^\mathrm{Re})}^T\boldsymbol{\psi}_i)N^\mathrm{Im}_j+
\nabla\cdot({(\mathbb{A}^\mathrm{Im})}^T\boldsymbol{\psi}_i)N^\mathrm{Re}_j
\,\mathrm{d}\mathbf{x}\\
&=
i\int_{\Gamma_D}\left(A^\mathrm{Re}_{M_2}({(\mathbb{A}^\mathrm{Im})}^T\boldsymbol{\psi}_i)+
A^\mathrm{Im}_{M_2}({(\mathbb{A}^\mathrm{Re})}^T\boldsymbol{\psi}_i)
\right)\cdot\mathbf{n}\,\mathrm{d}s
+
i\int_{\Gamma_N}\left(N^\mathrm{Re}_j({(\mathbb{A}^\mathrm{Im})}^T\boldsymbol{\psi}_i)+
N^\mathrm{Im}_j({(\mathbb{A}^\mathrm{Re})}^T\boldsymbol{\psi}_i)
\right)\cdot\mathbf{n}\,\mathrm{d}s
\end{align}
for all admissible pairs of real valued test functions $(\varphi_i,\boldsymbol{\psi}_i)\in W\times\mathbf{Q}_0$.

Let us define the following auxiliary matrices and vectors
\begin{alignat}{2}
A&=\{a_{ij}\} \qquad a_{ij}&=&\int_\Omega \psi_i\psi_j\,\mathrm{d}\mathbf{x}\\
B_x&=\{b_{ij,x}\} \qquad b_{ij,x}&=&\int_\Omega \partial_x\psi_i\varphi_j\,\mathrm{d}\mathbf{x}\\
B_y&=\{b_{ij,y}\} \qquad b_{ij,x}&=&\int_\Omega \partial_y\psi_i\varphi_j\,\mathrm{d}\mathbf{x}
\end{alignat}

\begin{equation}
\begin{bmatrix}
0 & -\omega M & B_x^T & 0 & B_y^T & 0\\
\omega M & 0 & 0 & B_x^T & 0 & B_y^T\\
B_x & 0 & A & 0 & 0 & 0\\
0 & B_x & 0 & A & 0 & 0\\
B_y & 0 & 0 & 0 & A & 0\\
0 & B_y & 0 & 0 & 0 & A
\end{bmatrix}
\begin{bmatrix}
N^\mathrm{Re}\\
N^\mathrm{Im}\\
\sigma_1^\mathrm{Re}\\
\sigma_1^\mathrm{Im}\\
\sigma_2^\mathrm{Re}\\
\sigma_2^\mathrm{Im}
\end{bmatrix}
=
\begin{bmatrix}
0\\
0\\
\phantom{i}\int_{\Gamma_D}A^\mathrm{Re}_{M_2}\partial_x\psi^x_i\,\mathrm{d}s\\
i\int_{\Gamma_D}A^\mathrm{Im}_{M_2}\partial_x\psi^x_i\,\mathrm{d}s\\
\phantom{i}\int_{\Gamma_D}A^\mathrm{Re}_{M_2}\partial_y\psi^y_i\,\mathrm{d}s\\
i\int_{\Gamma_D}A^\mathrm{Im}_{M_2}\partial_y\psi^y_i\,\mathrm{d}s
\end{bmatrix}
\end{equation}

\paragraph{Second formulation.} The problem reads: find $(N^\mathrm{Re},N^\mathrm{Im},\boldsymbol{\tau}^\mathrm{Re},\boldsymbol{\tau}^\mathrm{Im})\in [W]^2\times [\mathbf{Q}_0]^2$ such that
\begin{align}
&\sum_j\int_\Omega \varphi_i\left(\nabla\cdot(\mathbb{A}^\mathrm{Re}\boldsymbol{\tau}_j^\mathrm{Re}-\mathbb{A}^\mathrm{Im}\boldsymbol{\tau}_j^\mathrm{Im})-\omega N^\mathrm{Im}_j\right)+
\boldsymbol{\psi}_i\cdot\boldsymbol{\tau}^\mathrm{Re}_j+
\nabla\cdot\boldsymbol{\psi}_iN^\mathrm{Re}_j
\,\mathrm{d}\mathbf{x}\\
&=
\int_{\Gamma_D}(\boldsymbol{\psi}_i\cdot\mathbf{n})A^\mathrm{Re}_{M_2}\,\mathrm{d}s
+
\sum_j\int_{\Gamma_N}(\boldsymbol{\psi}_i\cdot\mathbf{n})N^\mathrm{Re}_j\,\mathrm{d}s
\end{align}
and
\begin{align}
&\sum_ji\int_\Omega \varphi_i\left(\nabla\cdot(\mathbb{A}^\mathrm{Re}\boldsymbol{\tau}_j^\mathrm{Im}+\mathbb{A}^\mathrm{Im}\boldsymbol{\tau}_j^\mathrm{Re})+\omega N^\mathrm{Re}_j\right)+
\boldsymbol{\psi}_i\cdot\boldsymbol{\tau}^\mathrm{Im}_j+
\nabla\cdot\boldsymbol{\psi}_iN^\mathrm{Im}_j
\,\mathrm{d}\mathbf{x}\\
&=
i\int_{\Gamma_D}(\boldsymbol{\psi}_i\cdot\mathbf{n})A^\mathrm{Im}_{M_2}\,\mathrm{d}s
+
\sum_ji\int_{\Gamma_N}(\boldsymbol{\psi}_i\cdot\mathbf{n})N^\mathrm{Im}_j\,\mathrm{d}s
\end{align}
for all admissible pairs of real valued test functions $(\varphi_i,\boldsymbol{\psi}_i)\in W\times\mathbf{Q}_0$.
In addition to the mass matrices $A$ and $M$ introduced above, let us define the following auxiliary matrices and vectors
\begin{alignat}{2}
C^\mathrm{Re}_k&=\{c^\mathrm{Re}_{ij,k}\} \qquad c^\mathrm{Re}_{ij,k}&=&
\int_\Omega \varphi_i\nabla\cdot(\mathbf{a}^\mathrm{Re}_k\psi_j)\,\mathrm{d}\mathbf{x}
=
\int_\Omega \varphi_i\left(\partial_x(a^\mathrm{Re}_{1k}\psi_j)+\partial_y(a^\mathrm{Re}_{2k}\psi_j)\right)\,\mathrm{d}\mathbf{x}\\
C^\mathrm{Im}_k&=\{c^\mathrm{Im}_{ij,k}\} \qquad c^\mathrm{Im}_{ij,k}&=&
\int_\Omega \varphi_i\nabla\cdot(\mathbf{a}^\mathrm{Im}_k\psi_j)\,\mathrm{d}\mathbf{x}
=
\int_\Omega \varphi_i\left(\partial_x(a^\mathrm{Im}_{1k}\psi_j)+\partial_y(a^\mathrm{Im}_{2k}\psi_j)\right)\,\mathrm{d}\mathbf{x}\\
D_k&=\{d_{ij,k}\} \qquad d_{ij,k}&=&
\int_{\Gamma_N}\psi_i\varphi_jn_k\,\mathrm{d}s\\
b^\mathrm{Re}_k&=\{b^\mathrm{Re}_{i,k}\} \qquad b^\mathrm{Re}_{i,k}&=&
\int_{\Gamma_D}A^\mathrm{Re}_{M_2}\psi_in_k\,\mathrm{d}s\\
b^\mathrm{Im}_k&=\{b^\mathrm{Im}_{i,k}\} \qquad b^\mathrm{Im}_{i,k}&=&
\int_{\Gamma_D}A^\mathrm{Im}_{M_2}\psi_in_k\,\mathrm{d}s
\end{alignat}
where $\mathbf{a}^\mathrm{Re}_k$ and $\mathbf{a}^\mathrm{Im}_k$ denotes the real and imaginary parts of the $k$-th column of matrix $\mathbb{A}$ and $n_k$ represents the $k$-th component of the outward unit normal vector $\mathbf{n}=(n_1,n_2)$. 

Then, the matrix form reads
\begin{equation}
\begin{bmatrix}
0 & -\omega M & C_1^\mathrm{Re} & -C_1^\mathrm{Im} & C_2^\mathrm{Re} & -C_2^\mathrm{Im}\\[1ex]
\omega M & 0 & C_1^\mathrm{Im} & C_1^\mathrm{Re} & C_2^\mathrm{Im} & C_2^\mathrm{Re}\\[1ex]
B_x+D_1 & 0 & A & 0 & 0 & 0\\[1ex]
0 & B_x+D_1 & 0 & A & 0 & 0\\[1ex]
B_y+D_2 & 0 & 0 & 0 & A & 0\\[1ex]
0 & B_y+D_2 & 0 & 0 & 0 & A
\end{bmatrix}
\begin{bmatrix}
N^\mathrm{Re}\\[1ex]
N^\mathrm{Im}\\[1ex]
\tau_1^\mathrm{Re}\\[1ex]
\tau_1^\mathrm{Im}\\[1ex]
\tau_2^\mathrm{Re}\\[1ex]
\tau_2^\mathrm{Im}
\end{bmatrix}
=
\begin{bmatrix}
0\\[1ex]
0\\[1ex]
b_1^\mathrm{Re}\\[1ex]
b_1^\mathrm{Im}\\[1ex]
b_2^\mathrm{Re}\\[1ex]
b_2^\mathrm{Im}
\end{bmatrix}
\end{equation}

\bibliographystyle{apalike}
\bibliography{mybib.bib}
\end{document}