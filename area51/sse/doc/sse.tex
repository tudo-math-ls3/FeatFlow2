\documentclass[11pt,a4paper]{article}
\usepackage[utf8]{inputenc}
\usepackage[english]{babel}
\usepackage{amsmath}
\usepackage{amsfonts}
\usepackage{amssymb}
\usepackage[left=2cm,right=2cm,top=2cm,bottom=2cm]{geometry}
\title{Elliptic equation of sea surface elevation}
\author{Matthias M\"oller}
\begin{document}
\maketitle

\section*{Model problem}
The elliptic equation for sea surface elevation is given by
\begin{alignat}{2}
-\nabla\cdot(A\nabla N)-i\omega N&=0  \quad && \text{in } \Omega,
\label{eq:sse_scalar1}\\
N&=A_{M_2} \quad && \text{on } \Gamma_D,
\label{eq:sse_scalar2}\\
-(A\nabla N)\cdot\mathbf{n}&=0 \quad && \text{on } \Gamma_N,
\label{eq:sse_scatar3}
\end{alignat}
where $\Gamma_D$ and $\Gamma_N$ denote the Dirichlet and Neumann boundary part and $A_{M_2}$ is the amplitude of sea surface elevation at the Dirichlet boundary. Both the solution $N\in\mathbb{C}$ and the matrix
$$
A=-
\begin{pmatrix}
C_1 & C_2\\
C_3 & C_4
\end{pmatrix},\quad\text{where}\quad C_k\in\mathbb{C}, k=1,\dots,4
$$
consist of complex valued data. Here, the coefficients $C_1,\dots,C_4$ are complex valued functions of horizontal coordinates. Let us define
\begin{align*}
c_{11}&=\frac{g}{r_1^3A_\nu}\left[\frac{s \sinh r_1 h}{r_1 A_\nu \sinh r_1 h + s \cosh r_1 h}-r_1 h\right],\\
c_{12}&=ic_{11},\\
c_{13}&=\frac{g}{r_2^3A_\nu}\left[\frac{s \sinh r_2 h}{r_1 A_\nu \sinh r_2 h + s \cosh r_2 h}-r_2 h\right],\\
c_{14}&=-ic_{13},
\end{align*}
where $i$ denotes the imaginary number, i.e. $i^2=-1$. Then, the auxiliary coefficients $C_1,\dots,C_4$ are defined as follows:
\begin{align}
C_1&=\frac{c_{11}+c_{13}}{2},\\
C_2&=\frac{c_{12}+c_{14}}{2},\\
C_3&=\frac{c_{11}-c_{13}}{2},\\
C_4&=\frac{c_{12}-c_{14}}{2}.
\end{align}
The other parameters are given in the following table

\begin{center}
\begin{tabular}{|l|l|l|}
\hline
notation & value & description\\
\hline
$\omega$ & $1.4\times 10^{-4}$ &tidal frequency\\
$g$ & 10 & gravitational acceleration\\
$s$ & 0.049 & bottom stress\\
$h$ & 10 & bottom profile\\
$A_\nu$ & 0.012 & eddy viscosity\\
$f$ & 0 & Coriolis acceleration (Earth rotation)\\
\hline
\end{tabular}
\end{center}
Moreover, the auxiliary parameters $r_1=\sqrt{i\frac{f-\omega}{A_\nu}}$ and $r_2=\sqrt{-i\frac{f+\omega}{A_\nu}}$ are adopted.

The variational formulation reads: find $N\in V_{A_{M_2}}$ such that
\begin{equation}
\int_\Omega(A\nabla N)\cdot\nabla \varphi- i\omega N \varphi\,\mathrm{d}\mathbf{x}=0\quad \forall \varphi\in V_0,
\end{equation}
The essential boundary conditions have been built into the trial and test spaces
\begin{align}
V_{A_{M_2}}&=\{v\in H_1(\Omega)\,:\, v=A_{M_2}\text{ on }\Gamma_D\},\\
V_0&=\{\varphi\in H_1(\Omega)\,:\, \varphi=0\text{ on }\Gamma_D\}.
\end{align}
Let the solution be approximated by finite elements as follows
\begin{equation}
N(\mathbf{x})=\sum_{j}\varphi_j(\mathbf{x})N_j 
\end{equation}
where $\{\varphi_j(\mathbf{x})\}_j$ denotes a basis of \emph{real} valued functions and complex valued coefficients $N_j\in\mathbb{C}$.

Instead of working with complex values directly we split them into their real and imaginary parts, e.g., $N_j=N_j^\textrm{Re}+iN_j^\textrm{Im}$ and use separate scalar finite element fields $N^\textrm{Re}$ and $N^\textrm{Im}$ for discretizing each one of them. Then the problem at hand reads: find $(N^\textrm{Re},N^\textrm{Im})\in V_{A_{M_2}^\textrm{Re}}\times V_{A_{M_2}^\textrm{Im}}$ such that
\begin{equation}
\sum_j\int_\Omega \nabla\varphi_i\cdot\left(A^\textrm{Re}\nabla N_j^\textrm{Re}-A^\textrm{Im}\nabla N_j^\textrm{Im}\right)+\omega \varphi_i N_j^\textrm{Im}\,\mathrm{d}\mathbf{x}=0
\end{equation}
and
\begin{equation}
\sum_ji\int_\Omega \nabla\varphi_i\cdot\left(A^\textrm{Im}\nabla N_j^\textrm{Re}+A^\textrm{Re}\nabla N_j^\textrm{Im}\right)-\omega \varphi_i N_j^\textrm{Re}\,\mathrm{d}\mathbf{x}=0
\end{equation}
for all admissible real valued test functions $\varphi_i\in V_{0}$.
To simplify the notation let us define the following auxiliary stiffness and mass matrices
\begin{alignat}{2}
S^\textrm{Re}&=\{s_{ij}^\textrm{Re}\} \qquad s_{ij}^\textrm{Re}&=&\int_\Omega \nabla\varphi_i\cdot(A^\textrm{Re}\nabla\varphi_j)\,\mathrm{d}\mathbf{x}\\
S^\textrm{Im}&=\{s_{ij}^\textrm{Im}\} \qquad s_{ij}^\textrm{Im}&=&\int_\Omega \nabla\varphi_i\cdot(A^\textrm{Im}\nabla\varphi_j)\,\mathrm{d}\mathbf{x}\\
M&=\{m_{ij}\} \qquad m_{ij}&=&\int_\Omega \varphi^\text{trial}\varphi^\text{test}\,\mathrm{d}\mathbf{x}
\end{alignat}
Then the problem at hand reads
\begin{equation}
\begin{pmatrix}
S^\textrm{Re} & -S^\textrm{Im}+\omega M\\
S^\textrm{Im}-\omega M &  S^\textrm{Re}
\end{pmatrix}
\begin{pmatrix}
N^\textrm{Re}\\
N^\textrm{Im}
\end{pmatrix}
=
\begin{pmatrix}
0\\
0
\end{pmatrix}
\end{equation}

\section*{First-order system}
Let $\boldsymbol{\sigma}=A\nabla N$ in order to convert the scalar problem equation\eqref{eq:sse_scalar1}--\eqref{eq:sse_scatar3} into the first-order system
\begin{alignat}{2}
\nabla\cdot\boldsymbol{\sigma}+i\omega N&=0  \quad && \text{in } \Omega,
\label{eq:sse_system1}\\
\boldsymbol{\sigma}-\nabla N&=0 \quad && \text{in } \Omega,\\
N&=A_{M_2} \quad && \text{on } \Gamma_D,
\label{eq:sse_system3}\\
\boldsymbol{\sigma}\cdot\mathbf{n}&=0 \quad && \text{on } \Gamma_N.
\label{eq:sse_system4}
\end{alignat}
The associated variational formulation reads: find $N\in V$ and $\boldsymbol{\sigma}\in\mathbf{W}_0$ such that 
\begin{alignat}{2}
\int_\Omega\left(\nabla\cdot\boldsymbol{\sigma}+i\omega N\right)\varphi\,\mathrm{d}\mathbf{x}&=0 \quad && \forall \varphi\in V,
\label{eq:weak1}\\
\int_\Omega\left(\boldsymbol{\sigma}-\nabla N\right)\cdot\boldsymbol{\psi}\,\mathrm{d}\mathbf{x}&=0 \quad && \forall \boldsymbol{\psi}\in \mathbf{W}_0,
\label{eq:weak2}
\end{alignat}
where $V=L^2(\Omega)$ and $\mathbf{W}_0=\{\boldsymbol{\psi}\in H(\text{div})\,:\,\boldsymbol{\psi}\cdot\mathbf{n}=0 \text{ on }\Gamma_N\}$.
Integration by parts is applied to \eqref{eq:weak2} to include the Dirichlet boundary condition \eqref{eq:sse_system3} as a natural one
\begin{equation}
\int_\Omega\boldsymbol{\sigma}\cdot\boldsymbol{\psi}+N\nabla\cdot\boldsymbol{\psi}\,\mathrm{d}\mathbf{x}=\int_{\Gamma_D} A_{M_2}\boldsymbol{\psi}\cdot\mathbf{n}\,\mathrm{d}s \quad \forall \boldsymbol{\psi}\in \mathbf{W}
\end{equation}
Thus, the variational formulation for the problem at hand reads: find $(N,\boldsymbol{\sigma})\in V\times\mathbf{W}_0$ such that
\begin{equation}
\int_\Omega\left(\nabla\cdot\boldsymbol{\sigma}+i\omega N\right)\varphi+\boldsymbol{\sigma}\cdot\boldsymbol{\psi}+N\nabla\cdot\boldsymbol{\psi}\,\mathrm{d}\mathbf{x}=\int_{\Gamma_D} A_{M_2}\boldsymbol{\psi}\cdot\mathbf{n}\,\mathrm{d}s \quad \forall (\varphi,\boldsymbol{\psi})\in V\times\mathbf{W}
\end{equation}
Separating the real and imaginary parts as in the scalar case the problem at hand reads: find $(N^\textrm{Re},N^\textrm{Im},\boldsymbol{\sigma}^\textrm{Re},\boldsymbol{\sigma}^\textrm{Im})\in [V]^2\times [\mathbf{W}_0]^2$ such that
\begin{equation}
\sum_j\int_\Omega \varphi_i\left(\nabla\cdot\boldsymbol{\sigma}_j^\textrm{Re}-\omega N^\textrm{Im}_j\right)+
\boldsymbol{\psi}_i\cdot\boldsymbol{\sigma}^\textrm{Re}_j+
\nabla\cdot\boldsymbol{\psi_i}N_j^\textrm{Re}\,\mathrm{d}\mathbf{x}=
\int_{\Gamma_D}(\boldsymbol{\psi}_i\cdot\mathbf{n})A^\textrm{Re}_{M_2}\,\mathrm{d}s
\end{equation}
and
\begin{equation}
\sum_ji\int_\Omega \varphi_i\left(\nabla\cdot\boldsymbol{\sigma}_j^\textrm{Im}+\omega N^\textrm{Re}_j\right)+
\boldsymbol{\psi}_i\cdot\boldsymbol{\sigma}^\textrm{Im}_j+
\nabla\cdot\boldsymbol{\psi_i}N_j^\textrm{Im}\,\mathrm{d}\mathbf{x}=
i\int_{\Gamma_D}(\boldsymbol{\psi}_i\cdot\mathbf{n})A^\textrm{Im}_{M_2}\,\mathrm{d}s
\end{equation}
for all admissible pairs of real valued test functions $(\varphi_i,\boldsymbol{\psi}_i)\in V_{0}\times\mathbf{W}_0$.
Let us define the following auxiliary matrices
\begin{alignat}{2}
A^x&=\{a^x_{ij}\} \qquad a^x_{ij}&=&\int_\Omega \psi^x_i\psi^x_j\,\mathrm{d}\mathbf{x}\\
A^y&=\{a^y_{ij}\} \qquad a^y_{ij}&=&\int_\Omega \psi^y_i\psi^y_j\,\mathrm{d}\mathbf{x}\\
B^x&=\{b^x_{ij}\} \qquad b^x_{ij}&=&\int_\Omega \partial_x\psi^x_i\varphi_j\,\mathrm{d}\mathbf{x}\\
B^y&=\{b^y_{ij}\} \qquad b^y_{ij}&=&\int_\Omega \partial_y\psi^y_i\varphi_j\,\mathrm{d}\mathbf{x}\\
C^x&=\{c^x_{ij}\} \qquad c^x_{ij}&=&\int_\Omega \varphi_i\partial_x\psi^x_j\,\mathrm{d}\mathbf{x}\\
C^y&=\{c^y_{ij}\} \qquad c^y_{ij}&=&\int_\Omega \varphi_i\partial_y\psi^y_j\,\mathrm{d}\mathbf{x}
\end{alignat}

\begin{equation}
\begin{pmatrix}
0 & -\omega M & C^x & 0 & C^y & 0\\
\omega M & 0 & 0 & C^x & 0 & C^y\\
B^x & 0 & A^x & 0 & 0 & 0\\
0 & B^x & 0 & A^x & 0 & 0\\
B^y & 0 & 0 & 0 & A^y & 0\\
0 & B^y & 0 & 0 & 0 & A^y
\end{pmatrix}
\begin{pmatrix}
N^\textrm{Re}\\
N^\textrm{Im}\\
\sigma_x^\textrm{Re}\\
\sigma_x^\textrm{Im}\\
\sigma_y^\textrm{Re}\\
\sigma_y^\textrm{Im}
\end{pmatrix}
=
\begin{pmatrix}
0\\
0\\
\phantom{i}\int_{\Gamma_D}A^\textrm{Re}\partial_x\psi^x_i\,\mathrm{d}s\\
i\int_{\Gamma_D}A^\textrm{Im}\partial_x\psi^x_i\,\mathrm{d}s\\
\phantom{i}\int_{\Gamma_D}A^\textrm{Re}\partial_y\psi^y_i\,\mathrm{d}s\\
i\int_{\Gamma_D}A^\textrm{Im}\partial_y\psi^y_i\,\mathrm{d}s
\end{pmatrix}
\end{equation}

\end{document}